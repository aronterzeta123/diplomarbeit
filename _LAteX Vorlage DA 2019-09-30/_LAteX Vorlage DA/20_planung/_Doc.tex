\chapter{\docname}
\label{\docname}
Ziele sind wesentlich für jedes Projekt. Deshalb wurden die Ziele dieses Projekts in drei Kategorien geteilt.
In der ersten Kategorie gehören Ziele, die unbedingt erfüllt werden müssen. Anderfalls wurde das Projekt scheitern.
In der zweiten Kategorie gehören Ziele, die optional sind. Das heißt sie sind nicht zwingend und wurden eingesetzt nur nachdem alle wichtigen und primären Ziele erfüllt sind.
Letztens sind die Nicht-Ziele definiert, damit das Projekt begrenzt ist und damit nichts gemacht wird, was nicht angefordert war.
\section{Projektziele}
Ziele, nicht Ziele und optionale Ziele
\subsection{Ziele}
\begin{enumerate}
	\item Live vs. Foto unterscheiden
	\item Gesichts-Schlüsselpunkt-Extraktion, um ein Gesicht zu identifizieren  

\item Größe und Form der Augenhöhlen, Nase, Wangenknochen und Kiefer analysieren 

\item Position/Verhältnisse der Hauptmerkmale relativ zueinander herausholen, 

\item Bilderdaten in Vektoren umwandeln mithilfe eines Algorithmus. 

\item Abstimmung (Vergleichen mit den anderen Fotos in der Datenbank, um zu sehen, ob die Person schon registriert wurde). 

\item Max. 500 Personen in einer Datenbank speichern

\item 10 Tests, jeder Test in einer anderen Raumkondition, um    alle Betriebskonditionen zu testen. 

\item Datenbankdesign 

\item Error checking

\item Safe Mode (eine Batterie, Back-ups in einem lokalen         Server) 

\item Min. Arbeitsvorbereitung (MIn. Gesichtsdetektionszeit)
\item Admin account(Register-Rechte nur für Schüler und Lehrer eingeben) 

\end{enumerate}
\subsection{nicht Ziele}
\begin{enumerate}
	\item Mehr als ein Gesicht gleichzeitig erkennen
	
	\item Maske, Brille, Hüte tragen 
	
	\item Gesicht in Bewegung erkennen 
	
	\item Person ins Profil oder andere Position sein
	
	\item Thermische Kamera einsetzen
	
	
\end{enumerate}
\subsection{optionale Ziele}
\begin{enumerate}
	
	
	\item Öffnung der Haustüren oder jeder anderen Tür mit Gesichtserkennung.
	
	\item LCD-Display Implementation
	
	\item Integration in dem Infotainment-System
	
	\item Licht neben der Kamera 
\end{enumerate}
\section{Projektplanung}
Unsere Big Picture ist unser erstes grobes Design, das die Lösungsskizze des Projekts beschreibt. Es gibt bestimmte Gründe, warum Big Picture und Structed Design verwendet wurden, um die Software zu beschreiben. Diese Methode ermöglicht eine sehr gute Darstellung und Beschreibung des Lösungswegs. Ist schnell und leicht zu machbar. Alles ist klar sichtbar und nicht kompliziert. Big Picture und Structed Design folgt das Top-Down Prinzip, das heißt die Funktionen werden hierarchisch zerlegt (Jede Funktion wird in die folgenden Ebenen detaillierter beschreibt). Structed Design und Big Picture haben keine Begrenzung. Dort können eindeutig alle Funktionen, Schnittstellen, Signalen und Daten beschreibt werden, sodass von allem leicht zu verstehen ist. 


\includegraphics[width=0.8\textwidth]{./figures/scrum.jpg}

\section{Projektmanagementmethode}
Als Projektplanmethode haben wir Scrum, eine agile Methode, gewählt, weil es die Möglichkeit bietet, komplexe Projekte mit einem kleinen Personenkreis zu verwalten. Scrum ist ideal für Software- bzw. Hardware-Entwicklungsteams, weil das Team während des Projekts verschiedene Änderungen an seinem Plan vornehmen muss. Aus diesem Grund ist es besser, tägliche Zielvorgaben zu haben und in einem kurzen Zeitraum von 1 bis 4 Wochen so genannte Sprints durchzuführen, bei denen das Ziel am Ende dieser Springs ein Prototyp ist. Verschiedene Prototypen herzustellen und am Ende den richtigen auszuwählen, ist die beste Wahl für die Projektmanagementmethode zur Gesichtserkennung. Es gibt auch tägliche Pläne, in denen sich das Team zusammensetzt und entscheidet, was die Ziele für den Tag sind und was sie tun müssen. 


\includegraphics[width=0.8\textwidth]{./figures/Big_Picture.jpg}

