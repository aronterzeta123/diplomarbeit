\chapter{\docname}
\label{\docname}

\color{black}
Unten werden die Idee, das Thema und die Aufgabenstellung dieser Diplomarbeit verfasst.
\begin{flushleft}
	Die Idee, ein System zu entwickeln dass Gesichter erkennt und registriert, ist daraus entstanden wegen folgenden Grunds. 
	Es wurde vom Auftraggeber dieses System gefordert, um einen kontrollierten Zugang in der Schule zu ermöglichen. Es ist gedacht, die Sicherheit der Schule dadurch zu erhöhen und die Überwachung effizienter machen. 
	Hauptziel ist es, alle Gesichter von den aktuellen Schülern und Lehrer zu registrieren und zu erkennen. 
	Das System sollte auch den Unterschied zwischen einer reellen Person und einem Foto berücksichtigen.  Es ist auch gefordert, dass die betreffende Person keine Maske, Brille oder Hüte bei der Gesichtserkennung trägt. Die Erkennung von Gesicht erfolgt auch nicht beim Bewegen von Person.
	Das Team besteht aus Aron Terzeta, Egli Hasmegaj, Rei Hoxha und Jordi Zmiani. 
	Aufgaben sind wie folgend geteilt. 
\end{flushleft}
\begin{itemize}
	
	
	
	\item Aron beschäftigt sich hauptsächlich mit der Gesichtsregistrierungsteil und Tiefenschärfe des Bildes herauszuholen. 
	\item Egli kümmert sich um die wichtigsten Gesicht Daten zu extrahieren(Größe und Form der Augenhöhlen, Nase, Wangenknochen und Kiefer). Position/ Verhältnisse der Hauptmerkmale relativ zueinander herausholen.  Aufbereitung der Daten für Abgleich.
	\item Rei: User-Gesichtsdaten von Bildverarbeitung-Funktion holen, Vergleichen von Gesichtsdaten, System aufbauen.
	\item Jordi: Datenbankdesign: Eine DB einrichten, Entwurf der Struktur der DB, DB in MySQL implementieren, Zugriffsberechtigungen festlegen, Error-checking. 
	
	
	Wir sind dafür hoch motiviert, dieses Projekt richtig umzusetzen.
	
\end{itemize}
