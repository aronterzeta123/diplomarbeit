\chapter{\docname}
\label{\docname}

\color{black}
Unten werden die Idee, das Thema und die Aufgabenstellung dieser Diplomarbeit beschrieben.
\begin{flushleft}
	In der heutigen Zeit ist die Technologie eines der wichtigsten Themen für die Welt. Mithilfe der Technologie hat die Menschheit wesentliche Fortschritte im Bereich der Lebensqualitätsverbesserung und -erleichterung gemacht. Mehr als jemals zuvor kommen intelligente Systeme zum Einsatz, die den Arbeitsaufwand und die Arbeitskomplexität reduzieren. Diese Systeme werden in den wichtigsten Bereiche des Lebens verwendet, wie zum Beispiel: Menschlichen Beziehungen, Business, Transport, Banken, Medizin, Bildung und Kommunikation. Selbstverständlich gibt es zahlreiche Versuche von verschiedenen Menschen, auf diesen Systemen zuzugreifen. Deshalb ist die Sicherheit in diesen Systemen eine der aktuellsten und wichtigsten Themen. Der unkontrollierte Zugriff auf solche wichtigen und delikaten Systeme würde zu fatalen Konsequenzen führen. Aus diesem Grund muss sehr genau und detailliert geplant werden, welche Sicherheit Strategien die klügsten bzw. die Besten sind.   
	
	Das Hauptthema dieses Projekts ist genau die Sicherheit der Systeme, und zwar die Ermöglichung des kontrollierten Zugangs von Menschen in verschiedenen Gebäuden. Es wurde die Verantwortlichkeit übernommen, der Zugriff in der Österreichischen Schule “Peter Mahringer” sicherer zu machen. Zu diesem Zweck wird in diesem Projekt ein intelligentes System entwickelt, dass die Aufgabe hat, menschliche Gesichter (Gesichtern von Schülern, Lehrern, Eltern usw.) zu registrieren und zu erkennen. Es wird höchstwahrscheinlich in die Schultür integriert, weil dort alle Personen unbedingt vorbeigehen müssen. Dieses wird eine große Hilfe für die Schule sein, aufgrund der von Kameras und komplexen Algorithmen angebotenen Möglichkeiten. Beispielweise wird überprüft, ob die Person, die reinkommt ein Student oder Lehrer ist, oder es wird notiert, wer die Schule wann verlässt oder besucht. Hat eine Person kein Zugriff, wird sie nicht herein gelassen.
	
	Um dieses Projekt zu realisieren, wird das Projekt in drei große und wichtige Teile zerlegt. Das erste Teil heißt Bildverarbeitung und hat die Aufgabe, Gesichtsbildern von der Kamera zu normalisieren und zu verarbeiten, damit sie in geeigneten Format für den Gesichtsvergleichung sind. Der zweite Teil beschäftigt sich mit der Registrierung der Gesichter von Personen am Server. Der dritte Teil ist die Gesichtserkennung. Hier geht es um die Erkennung der Gesichter der Benutzer des Systems. 

	
	Das Team besteht aus Aron Terzeta, Egli Hasmegaj, Rei Hoxha und Jordi Zmiani. 
	Aufgaben sind wie folgend geteilt:
\end{flushleft}
\begin{itemize}
	
	\item Aron beschäftigt sich hauptsächlich mit dem Gesichtsregistrierungsteil. 
	\item Egli kümmert sich darum, die wichtigsten Gesicht Daten zu extrahieren(Größe und Form der Augenhöhlen, Nase, Wangenknochen und Kiefer). Position/ Verhältnisse der Hauptmerkmale relativ zueinander zu bestimmen. Aufbereitung der Daten für den Abgleich.
	\item Rei macht den Systemaufbau und die Gesichtserkennung.
	\item Jordi k\"ummert sich um die Datenbank und den 2D vs 3D Unterschied.

\end{itemize}
Wir sind dafür hoch motiviert, dieses Projekt richtig umzusetzen.