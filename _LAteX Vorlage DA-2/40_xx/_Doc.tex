\chapter{\docname}
\label{\docname}

\section{Leben}

Dirac wurde in Bristol, Gloucestershire, England geboren. Sein Vater Charles Dirac war Schweizer mit Wurzeln im französischsprachigen Saint-Maurice im Wallis; er unterrichtete in Bristol an Diracs Schule das Fach Französisch. Seine Mutter, Florence Holten, war die Tochter eines Seemanns aus Cornwall. Seine Kindheit war infolge des strengen und autoritären Verhaltens des Vaters unglücklich – ein Bruder nahm sich das Leben.

Dirac studierte zunächst 1921 Elektrotechnik in Bristol, wechselte dann zur Mathematik und bekam 1923 ein Stipendium für die Universität Cambridge, wo er bei Ralph Howard Fowler studierte. 1926 schloss er das Studium mit einer Dissertation zur Quantenmechanik ab.
Paul Dirac mit seiner Frau Margit
Juli 1963 in Kopenhagen

Von 1932 bis 1969 war Dirac Professor des Lucasischen Lehrstuhls für Mathematik an der Universität Cambridge. 1937 heiratete er Margit (1904–2002), die Schwester des Physikers Eugene Wigner. Der Mathematiker Gabriel Andrew Dirac aus der ersten Ehe seiner Frau war sein Stiefsohn. Während des Zweiten Weltkriegs arbeitete Dirac an Gaszentrifugen zur Urananreicherung. Ab 1970 war er an der Florida State University in Tallahassee in Florida tätig.

Dirac war von zurückhaltender Natur. Es machte ihm nichts aus, in Gesellschaft zu schweigen und auf Fragen nur sehr wortkarge, einer strikten Wahrheitsliebe verpflichtete Antworten zu geben, wovon zahlreiche Anekdoten verbreitet waren.

Dirac war überzeugter Atheist. Auf die Frage nach seiner Meinung zu Diracs Ansichten bemerkte Wolfgang Pauli in Anspielung auf das islamische Gottesbekenntnis:

„Wenn ich Dirac richtig verstehe, meint er Folgendes: Es gibt keinen Gott und Dirac ist sein Prophet.“





\section{Leistung}

1925 fand Paul Dirac in seiner Dissertation die klassische Entsprechung der neuen quantenmechanischen Kommutatoren von Heisenberg, Born und Jordan mit den Poisson-Klammern der klassischen Mechanik. 1926 entwickelte er eine abstrakte Fassung der Quantenmechanik („Transformationstheorie“), die die Matrizenmechanik Heisenbergs und die Wellenmechanik Schrödingers als Spezialfälle enthielt. Somit konnte er unabhängig von Schrödinger die Äquivalenz beider Theorien zeigen. Die klassische Mechanik ergibt sich in seiner Theorie als Spezialfall der Quantenmechanik. Von Dirac stammt auch die Einführung des Wechselwirkungsbilds, das sowohl das Schrödinger- als auch Heisenberg-Bild verwendet.
Paul Dirac an der Tafel

1928 stellte er auf Grundlage der Arbeit von Wolfgang Pauli über das Ausschließungsprinzip die nach ihm benannte Dirac-Gleichung auf,[1] bei der es sich um eine relativistische, also auf der speziellen Relativitätstheorie beruhende Wellengleichung 1. Ordnung zur Beschreibung des Elektrons handelt. Dirac fand sie, indem er von der relativistischen Wellengleichung 2. Ordnung von Charles Galton Darwin ausging (einer Weiterentwicklung der Klein-Gordon-Gleichung) und ein wenig mit „Gleichungen herumspielte“, das heißt, er suchte einen Ansatz für eine entsprechende Gleichung 1. Ordnung, die sich nur mit dem Einführen von Spinoren und Dirac-Matrizen gewinnen ließ und deren „Quadrat“ wieder die relativistische Wellengleichung ergibt. Sie lieferte z. B. eine theoretische Erklärung für den anomalen Zeeman-Effekt und die Feinstruktur in der Atomspektroskopie und erklärte den Spin, der bis dahin in der Quantenmechanik als grundlegendes, aber unverstandenes Phänomen bekannt war, als natürliche Folge seiner relativistischen Wellengleichung.

Seine Gleichung erlaubte es Dirac auch, die Löchertheorie zu formulieren und die Existenz des Positrons, des Antiteilchens des Elektrons, vorherzusagen (er scheute aber zunächst vor der öffentlichen Postulierung eines neuen Teilchens zurück und identifizierte das negative Antiteilchen des Elektrons mit dem Proton).[2] Das Positron wurde darauf 1932 von Carl David Anderson als neues Teilchen in kosmischer Strahlung nachgewiesen. Im Dirac-Bild der Quantenfeldtheorie besteht das Vakuum in Analogie zur Festkörperphysik aus einem bis zur Fermigrenze gefüllten Dirac-See von Elektronen. Paarerzeugung im Vakuum ist die Anregung eines Elektrons aus diesem Dirac-See über die Fermigrenze hinaus – das hinterlassene „Loch“ in dem Diracsee ist das Positron.



Dirac schuf den Begriff des Bosons in Anerkennung der Verdienste von Satyendra Nath Bose um die Quantenstatistik. Er gilt mit Enrico Fermi als Erfinder der Statistik der Fermionen (Fermi-Dirac-Statistik), erkannte aber Fermis Priorität an.

1931 postulierte er als erster die Existenz eines magnetischen Monopols,[3] also eines Teilchens mit magnetischer Ladung, ähnlich der elektrischen Ladung z. B. beim Elektron. Die Existenz eines solchen Teilchens, das bisher nicht beobachtet wurde, würde die Quantisierung der elektrischen Ladung erklären. Dahinter stecken letztlich topologische Ideen, die hier erstmals in der Quantenmechanik auftauchen.

In seiner „Large number hypothesis“ versucht Dirac – plausibler als ähnliche Versuche Eddingtons – einen Zusammenhang zwischen der Größe der Fundamentalkonstanten und der gegenwärtigen Ausdehnung des Universums zu geben.[4] Daraus ergeben sich Spekulationen über die zeitliche Variation der Naturkonstanten, denen bis heute experimentell nachgegangen wird. Diracs großer Konkurrent auf dem Gebiet quantenmechanischer Formalismen, Pascual Jordan, griff diese Ideen in einer eigenen Theorie der Gravitation mit variabler Gravitationskonstante auf.

In seiner Untersuchung der klassischen Theorie strahlender Elektronen von 1938 tauchten neben „runaway solutions“ auch erstmals Renormierungsideen auf.[5] Das Auftreten divergenter Ausdrücke in der üblichen Renormierungstheorie der Quantenelektrodynamik, die dann in die Definition der „nackten“ Ladung und Masse zum Verschwinden gebracht werden, lehnte er aber zeitlebens ab.
Paul Dirac, Wolfgang Pauli und Rudolf Peierls, 1953 in Birmingham

Dirac ist auch der Erfinder vieler weiterer Formalismen der theoretischen Physik. Beispielsweise stammt von ihm die ursprüngliche Idee zu Pfadintegralen,[6] die als alternativer Zugang zur Quantenmechanik aber erst durch Richard Feynman „ernst genommen“ und ausgebaut wurden. In einer Arbeit aus dem Jahre 1949 erfand er die „light cone quantization“ (Lichtfrontformalismus) der Quantenfeldtheorie,[7] die in der Hochenergiephysik viel verwendet wird. In den 1950er Jahren versuchte Dirac dann, den von ihm postulierten Dirac-See als universellen Äther auszulegen.[8][9][10]

Er untersuchte auch ganz allgemein hamiltonsche Systeme mit „constraints“ (Zwangsbedingungen), speziell um einen Zugang zur Quantisierung der Gravitation zu finden. Diese Arbeiten gingen später in der BRST-Formulierung auf. Seine Untersuchung ausgedehnter Systeme in der Quantenfeldtheorie 1962[11] ist ein Vorläufer der p-branes und bag-Modelle späterer Jahre. 
