\chapter{\docname}
\label{\docname}

\section{Etymology}
The origin of the name is unclear. Roman tradition maintained that it was related to Latin words connected to lightning (fulgur, fulgere, fulmen), which in turn was thought of as related to flames.[4] This interpretation is supported by Walter William Skeat in his etymological dictionary as meaning lustre.[5]

It has been supposed that his name was not Latin but related to that of the Cretan god Velchanos, a god of nature and the nether world.[6] Wolfgang Meid has dispued this identification as phantastic.[7] More recently this etymology has been taken up by Gérard Capdeville who finds a continuity between Cretan Minoan god Velchanos and Etruscan Velchans. The Minoan god's identity would be that of a young deity, master of fire and companion of the Great Goddess.[8]

Christian Guyonvarc'h has proposed the identification with the Irish name Olcan (Ogamic Ulccagni, in the genitive). Vasily Abaev compares it with the Ossetic Wærgon, a variant of the name of Kurdalægon, the smith of the Nart saga. Since the name in its normal form Kurdalægon is stable and has a clear meaning (kurd smith+ on of the family+ Alaeg name of one of the Nartic families), this hypothesis has been considered unacceptable by Dumezil.[9] 

\section{Worship}
Vulcan's oldest shrine in Rome, called the Vulcanal, was situated at the foot of the Capitoline in the Forum Romanum, and was reputed to date to the archaic period of the kings of Rome,[10][11] and to have been established on the site by Titus Tatius,[12] the Sabine co-king, with a traditional date in the 8th century BC. It was the view of the Etruscan haruspices that a temple of Vulcan should be located outside the city,[13] and the Vulcanal may originally have been on or outside the city limits before they expanded to include the Capitoline Hill.[1] The Volcanalia sacrifice was offered here to Vulcan, on August 23.[10] Vulcan also had a temple on the Campus Martius, which was in existence by 214 BC.[1][14]

The Romans identified Vulcan with the Greek smith-god Hephaestus.[15] Vulcan became associated like his Greek counterpart with the constructive use of fire in metalworking. A fragment of a Greek pot showing Hephaestus found at the Volcanal has been dated to the 6th century BC, suggesting that the two gods were already associated at this date.[11] However, Vulcan had a stronger association than Hephaestus with fire's destructive capacity, and a major concern of his worshippers was to encourage the god to avert harmful fires.

\section{Vulcanalia}

The festival of Vulcan, the Vulcanalia, was celebrated on August 23 each year, when the summer heat placed crops and granaries most at risk of burning.[1][16] During the festival bonfires were created in honour of the god, into which live fish or small animals were thrown as a sacrifice, to be consumed in the place of humans.[17]

The Vulcanalia was part of the cycle of the four festivities of the second half of August (Consualia on August 21, Vulcanalia on 23, Opiconsivia on 25 and Vulturnalia on 27) related to the agrarian activities of that month and in symmetric correlation with those of the second half of July (Lucaria on July 19 and 21, Neptunalia on 23 and Furrinalia on 25). While the festivals of July dealt with untamed nature (woods) and waters (superficial waters the Neptunalia and underground waters the Furrinalia) at a time of danger caused by their relative deficiency, those of August were devoted to the results of human endeavour on nature with the storing of harvested grain (Consualia) and their relationship to human society and regality (Opiconsivia) which at that time were at risk and required protection from the dangers of the excessive strength of the two elements of fire (Vulcanalia) and wind (Vulturnalia) reinforced by dryness.[18]

It is recorded that during the Vulcanalia people used to hang their clothes and fabrics under the sun.[19] This habit might reflect a theological connection between Vulcan and the divinized Sun.[20]

Another custom observed on this day required that one should start working by the light of a candle, probably to propitiate a beneficial use of fire by the god.[21] In addition to the Vulcanalia of August 23, the date of May 23, which was the second of the two annual Tubilustria or ceremonies for the purification of trumpets, was sacred to Vulcan.[16][22]

The Ludi Vulcanalici, were held just once on August 23, 20 BC, within the temple precinct of Vulcan, and used by Augustus to mark the treaty with Parthia and the return of the legionary standards that had been lost at the Battle of Carrhae in 53 BC.

A flamen, one of the flamines minors, named flamen Vulcanalis was in charge of the cult of the god. The flamen Vulcanalis officiated at a sacrifice to the goddess Maia, held every year at the Kalendae of May.[23]

Vulcan was among the gods placated after the Great Fire of Rome in AD 64.[24] In response to the same fire, Domitian (emperor 81–96) established a new altar to Vulcan on the Quirinal Hill. At the same time a red bull-calf and red boar were added to the sacrifices made on the Vulcanalia, at least in that region of the city.[25]
Andrea Mantegna: Parnas, Vulcan, god of fire


\section{Theology}

The nature of the god is connected with religious ideas concerning fire.

The Roman concept of the god seems to associate him to both the destructive and the fertilizing powers of fire.

In the first aspect he is worshipped in the Volcanalia to avert its potential danger to harvested wheat. His cult is located outside the boundaries of the original city to avoid the risk of fires caused by the god in the city itself.[26]

This power is, however, considered useful if directed against enemies and such a choice for the location of the god's cult could be interpreted in this way too. The same idea underlies the dedication of the arms of the defeated enemies,[27] as well as those of the surviving general in a devotion ritual to the god.[28]

Through comparative interpretation this aspect has been connected by Dumézil to the third or defensive fire in the theory of the three Vedic sacrificial fires.[29] In such theory three fires are necessary to the discharge of a religious ceremony: the hearth of the landlord, which has the function of establishing a referential on Earth in that precise location connecting it with Heaven; the sacrificial fire, which conveys the offer to Heaven; and the defensive fire, which is usually located on the southern boundary of the sacred space and has a protective function against evil influences. Since the territory of the city of Rome was seen as a magnified temple in itself, the three fires should be identified as the hearth of the landlord in the temple of Vesta (aedes Vestae); the sacrificial fires of each temple, shrine or altar; and the defensive fire in the temple of Vulcan.

Another meaning of Vulcan is related to male fertilizing power. In various Latin and Roman legends he is the father of famous characters, such as the founder of Praeneste Caeculus,[30] Cacus,[31] a primordial being or king, later transformed into a monster that inhabited the site of the Aventine in Rome, and Roman king Servius Tullius. In a variant of the story of the birth of Romulus the details are identical even though Vulcan is not explicitly mentioned.[32]
Punishment of Ixion: in the center is Mercury holding the caduceus and on the right Juno sits on her throne. Behind her Iris stands and gestures. On the left is Vulcanus (blond figure) standing behind the wheel, manning it, with Ixion already tied to it. Nephele sits at Mercury's feet; a Roman fresco from the eastern wall of the triclinium in the House of the Vettii, Pompeii, Fourth Style (60–79 AD).

Some scholars think that he might be the unknown god who impregnated goddesses Fortuna Primigenia at Praeneste and Feronia at Anxur. In this case he would be the father of Jupiter.[33] This view is though in conflict with that which links the goddess to Jupiter, as his daughter (puer Jovis) and his mother too, as primigenia, meaning "primordial".

In all of the above-mentioned stories the god's fertilizing power is related to that of the fire of the house hearth.

In the case of Caeculus, his mother was impregnated by a spark that dropped on her womb from the hearth while she was sitting nearby.[34] Servius Tullius's mother Ocresia was impregnated by a male sex organ that miraculously appeared in the ashes of the sacrificial ara, at the order of Tanaquil, Tarquinius Priscus's wife.[35] Pliny the Elder tells the same story, but states that the father was the Lar familiaris.[36] The divinity of the child was recognized when his head was surrounded by flames and he remained unharmed.[37]

Through the comparative analysis of these myths archaeologist Andrea Carandini opines that Cacus and Caca were the sons of Vulcan and of a local divine being or a virgin as in the case of Caeculus. Cacus and Caca would represent the metallurgic and the domestic fire, projections of Vulcan and of Vesta.

These legends date back to the time of preurban Latium. Their meaning is quite clear: at the divine level Vulcan impregnates a virgin goddess and generates Jupiter, the king of the gods; at the human level he impregnates a local virgin (perhaps of royal descent) and generates a king.[38]

The first mention of a ritual connection between Vulcan and Vesta is the lectisternium of 217 BC. Other facts that seem to hint at this connection are the relative proximity of the two sanctuaries and Dionysius of Halicarnassus's testimony that both cults had been introduced to Rome by Titus Tatius to comply with a vow he had made in battle.[39] Varro confirms the fact.[40]

Vulcan is related to two equally ancient female goddesses Stata Mater,[41] perhaps the goddess who stops fires and Maia.[42]

Herbert Jennings Rose interprets Maia as a goddess related to growth by connecting her name with IE root *MAG.[43] Macrobius relates Cincius's opinion that Vulcan's female companion is Maia. Cincius justifies his view on the grounds that the flamen Volcanalis sacrificed to her at the Kalendae of May. In Piso's view the companion of the god is Maiestas.[44]

According to Gellius as well, Maia was associated with Vulcan; and he backs up his view by quoting the ritual prayers in use by Roman priests.[45]

[46]

The god is the patron of trades related to ovens (cooks, bakers, confectioners) as attested in the works of Plautus,[47] Apuleius (the god is the cook at the wedding of Amor and Psyche)[48] and in Vespa's short poem in the Anthologia Latina about the litigation between a cook and a baker.[49] 

