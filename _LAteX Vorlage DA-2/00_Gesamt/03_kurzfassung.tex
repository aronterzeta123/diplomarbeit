\chapter*{Kurzfassung}

\begin{flushleft}
	Vom Auftraggeber wird ein System gefordert, das Gesichter erkennt, um einen kontrollierten Zugang in der Schule zu ermöglichen und die Sicherheit der Schule wird dadurch erhöht. Alle Gesichter sollen von den aktuellen Schülern und Lehrer erkannt werden. Es soll auch zwischen einer reellen Person und einem Foto den Unterschied berücksichtigt werden.
	
	Wir haben uns für diese Idee entschieden, weil Sicherheit heute hoch interessant und relevant ist. Es geht hier um einen kontrollierten Zugang in Institutionen mittels Gesichtserkennung zu ermöglichen, da Gesichter eindeutig für jede Person sind. Die größten Herausforderungen und Voraussetzungen des Projekts befinden sich in dem Planungsprozess. Eine andere Voraussetzung ist das Gebrauch von zwei Kameras, damit der Unterschied zwischen einer reellen Person und einem Foto berücksichtigt wird.
	
	
	
\end{flushleft}

\color{black} 
