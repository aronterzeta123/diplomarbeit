\chapter{\docname}
\label{\docname}



	Diese Diplomarbeit besteht aus vielen unterschiedlichen Modulen, die um
	bestimmte Ziele bzw. Aufgaben zu lösen gut aufgeteilt sind.
	Ein sehr wichtiger Teil dieses Projektes ist der Bildverarbeitungsteil.
	In dem folgenden Abschnitt der Ausarbeitung wird  genau erklärt und beschrieben
	wie diese Aufgabe gelöst wurde und was dafür verwendet wurde.


\section{Allgemeines}

In diesem Allgemeinen Abschnitt werden die verwendeten Technologien und
Bibliotheken beschrieben.

\subsection{Entwicklungsumgebung und Technologien}


	Die gewählte Entwicklungsumgebung war grundsätzlich eine virtuelle Maschine die
	Linux auf einem Windows System geboten hat. 
	Linux im Gegensatz zu Windows ermöglicht volle Kontrolle über Updates und
	Upgrades und dadurch könnten komplexe Aufgaben einfacher erlediget werden.
	Alles wird leicht und bequem durch die Konsole eingegeben. 
	So wurden die Entwicklung und das Testen der genutzten Algorithmen und andere
	Technologien erleichtert. \cite{linx}\\
	
	Zusätzlich ist Raspberry Pi als Backup System verwendet worden dies, funktioniert
	auch mit einem lauffähigen Linux Betriebssystem (Debian). 
	Der Raspberry Pi ist klein, funktioniert aber wie ein normaler Rechner und ist 
	kostengünstig. Für technische/elektronische Projekte wie das Betreffende, ist es
	perfekt geeignet. 
	Für die Implementierung des Codes wurde hauptsächlich die Programmiersprache
	Python verwendet. 
	Python ist eine höhere Programmiersprache. Dies bedeutet dass es näher an
	menschlichen Sprachen ist. 
	Also ist ein in Python geschriebener Code sehr leicht von einem Mensch zu lesen,
	zu verwalten und zu warten.  \\ 
	
	Es bietet zahlreiche Module, mit seiner großen und robusten Standardbibliothek
	an, von denen man beruhigt, abhängig vom Bedarf, auswählen kann.  
	Sehr leicht kann man in der Dokumentation der Python-Standardbibliothek nachsehen
	um sich besser mit den Funktionalitäten auszukennen. 
	\cite{why_python}
	
	
	"Git ist ein freies und Open Source verteiltes Versionskontrollsystem, das
	entwickelt wurde, um alles von kleinen bis zu sehr großen Projekten mit
	Geschwindigkeit und Effizienz abzuwickeln." \cite{Git1} \\
	Die Versionierung der Software Änderungen erfolgte durch Git.
	Es hat sich als nützlich herausgestellt, da die Arbeit dadurch zwischen die
	Projektmitgliedern sehr gut koordiniert und verwaltet wurde. 
	



\subsection{Frameworks und Bibliotheken}


	
	Das Schlüsselwort von unserem Projekt war das Framework bzw. die Bibliothek
	„OpenCV“.
	OpenCV ist eine open-source Bibliothek für Computer Vision. Ganz allgemein
	kann sie als eine Bibliothek für Bildverarbeitung betrachtet werden.
	Sie beschäftigt sich unter anderem mit der Manipulation und Analyse von Bildern,
	die Analyse von denen und um die Bestimmung von Muster bzw. Objekte, für
	verschiedenen Zwecken. Ganz berühmte Anwendungsgebieten sind
	Gesichtserkennung und Stereo Vision. 
	Stereo Vision bedeutet die Extrahierung von Informationen aus einem Bild in eine
	3-dimensionale Ebene. 
	OpenCV wurde unter anderen Bibliotheken aufgrund seiner vielen guten
	Eigenschaften und seiner Flexibilität ausgewählt. Es umfasst hunderte von
	Computer-Vision-Algorithmen und besteht aus strukturierten Einheiten bzw. Module
	die als feststehende Bibliotheken implementiert sind. 
	Es ist Cross-Platform, in C/C++ geschrieben und unterstützt auch Python. 
	\cite{opencv_library}
	
	SciPy hat sich als nützlich, beim Lösen von gewisse mathematische Angaben und
	bei zusätzliche Installationen von Bibliotheken herausgestellt. 
	"SciPy ist eine kostenlose, Open-Source-Python-Bibliothek für wissenschaftliches
	und technisches Rechnen."\cite{2019arXiv190710121V-scipy}
	\\ 
	Dies sind einige der Kernpakete von SciPy die für das Projekt verwendet
	werden.
	
	NumPy legt die Basis für das wissenschaftliche Rechnen mit Python.
	Es unterstützt große mehrdimensionale Arrays und Matrizen. Es enthält viele
	anspruchsvolle Mathematische Funktionen um die Arrays zu bearbeiten.
	NumPy kann also als leistungsfähiger mehrdimensionaler Behälter für generische
	Daten verwendet werden. Es können beliebige Datentypen definiert werden.
	
	\textbf{Dlib} ist ein so genanntes Toolkit, das Algorithmen für Machine Learning
	und Tools zum Erstellen komplexer Software zur Lösung von Probleme der realen
	Welt beinhaltet.
	Es wird vielseitig eingesetzt, in Robotik, bei Mobilgeräte und unter anderem in
	Computer Vision.\cite{dlib} 
	Es wurde bei der Extrahierung der Gesichtsmerkmalen gebraucht. Das Logo von Dlib ist auf das \ref{fig:dlib logo} ersichtlich. 
	\begin{figure}[H]
		\centering
		\includegraphics[scale=0.2]{\ordnerfigures dlib.png}
		\caption{dlib logo \cite{dlib}}
		\label{fig:dlib logo}
	\end{figure}
	


\section{Technische Lösungen}


	Bei der Gesichtsregistrierungs und Gesichtserkennungs Diplomarbeit war es eine
	schwierige und herausfordernde Aufgabe, sie sorgfältig zu planen, um die
	Effektivität bei der Arbeit zu steigern und das Endprodukt zufriedenstellend zu
	machen. 
	Wichtig war es die Ziele richtig zu setzen und sie gut abzugrenzen, damit es zu
	keinen Lücken bei der Implementierung kommt.
	Aus diesem Grund musste die Arbeitsteilung gut geregelt werden, damit jedes
	Teammitglied sich auf bestimmte Modul der Arbeit konzentrieren konnte.
	Es wurde auch berücksichtigt, dass jedes Teammitglied das machte was ihm/ihr am
	besten gefällt und was ihm/ihr am leichtesten fällt. 
	Die Planung der betreffenden Bereiche der Projektarbeit ist durch die Methode
	„Structed Design“ erfolgt. 
	Structed Design ist eine Erweiterung von der „Big Picture“ Methode, die dazu
	dient, ein technisches System mit ihren Schnittstellen von außen grob zu
	beschreiben. 
	Es ist in verschieden Ebenen unterteilt, von außen beginnend. 
	Nun wird die erste Ebene der Structed Design von der Bildverarbeitungsteil
	dargestellt. 

\subsection{Lösungsweg - Structed Software design} 


\begin{figure}
	\centering
	\includegraphics[width=\textwidth]{\ordnerfigures bildverarbeitung.png}
	\caption{ Bildverarbeitung Structed Design}
	\label{fig:bildverarbeitung}
\end{figure}
\subsubsection{Legende:}
\begin{figure}[H]
	\includegraphics[width=\textwidth]{\ordnerfigures Erklaerung_Erste_Ebene.png}
	\caption{Erste Ebene Erklärung}
	\label{fig:Erklaerung_Erste_Ebene}
\end{figure}

Wie es in der Abb.\ref{fig:bildverarbeitung} ersichtlich ist, ist der
Bildverarbeitungsteil in diese Einheiten unterteilt:
\begin{enumerate}
	\item Bildaufnahme
	\item Bild Normalisierung
	\item Gesichtschlüsselpunkten Extrahierung
	\item Log
	\item Output 
	\item Trigger 
\end{enumerate}


	Die Schnittstellen nach außen sind das Gesichtserkennungsmodul und das
	Gesichtsregistrierungsmodul.  
	Diese schicken ein Signal an den Trigger der dann das Bildaufnahmemodul
	initialisiert. Unabhängig von welchen dieser beiden Schnittstellen die Anfrage
	kommt, macht das Bildaufnahmemodul mit den zwei Kameras ein Foto. 
	\justify
	Implementierungsnah wurde nur eine Kamera verwendet, da die Stereo Vision
	Probleme gegeben hat. 
	Nachdem das Foto gemacht wird folgt die Image Normalisierung bzw. das „Image
	Processing“. 
	Wenn ein Bild zu klein ist, wird die Größe angepasst, wenn ein Gesicht verdreht
	ist wird es gerade rotiert. Mehr dazu wird im Normalisierungabschnitt erklärt.
	
	Danach folgt die Erkennung von Gesichtern, das Ausschneiden von denen und die
	Extrahierung der Gesischtsschlüsselpunkte. 
	Sie werden dann zum Abgleich oder Registrierung je nach Bedarf, geschickt.
	
	Bei der Umsetzung wurden bestimmte Bereiche miteinander verknüpft, was zu
	Folgenden führte: was bei der Planung steht, stimmt nicht vollig mit der
	Methoden zur Umsetzung überein.


\subsection{Gesichtserkennung}

	Das erste Ziel, dass bei der Gesichtserkennung erfüllt worden war ist das Finden
	von Gesichtern, also die Feststellung, wo sich die Gesichter im Bild befinden.
	 
	Dafür ist der vorgefertigter Klassifikator \textit{„Haar Cascade Classifiers“}
	benutzt worden.  
	Gesichtserkennung durch „Haar“ Merkmale ist eine sehr effektive Methode die von
	Paul Viola und Michael Jones entwickelt wurde, indem sie Merkmale gruppiert
	haben und die Erkennung von diesen dadurch schneller erfolgt ist.  
	Die Arbeit heißt „Rapid Object Detection using a Boosted Cascade of Simple
	Features". \cite{Viola01robustreal-time}
	
	
	Unten wird eine kurze technische Übersicht über diese Klassifikator gegeben.\\
	
	Es geht hier um Maschinelles Lernen, wie diese Kaskadenfunktionen durch viele
	negative(Bilder ohne Gesichter) und positive(Bilder mit Gesichter) Bilder
	trainiert wurden um Objekte zu erkennen. 
	In diesem Fall  arbeiten wir selbstverständlich mit Gesicht Objekten. 
	Dafür wurden Merkmale Haare benutzt. Auf Abbildung \ref{fig:haar features}
	ist dies dargestellt. Jedes Haar Merkmal ist nichts anders als ein Wert, den man
	durch das Subtrahieren der Summe der Pixel unter dem weißen Rechteck von der
	Summe der Pixel unter dem schwarzen Rechteck erhält. 
	%Diese Summen berechnet man durch integrale Bilder, die zur Vereinfachung zu
	%Summenberechnungen dienen. 
	\begin{figure}[H]
		\centering
		\includegraphics[scale=0.8]{\ordnerfigures haar_features.jpg}
		\caption{ Haar Features\cite{Viola01robustreal-time}}
		\label{fig:haar features}
	\end{figure}
	
	Man muss aufpassen, da nicht alle Merkmale von Nutzen sein könnten. Man kann zum
	Beispiel deutlich auf Abbildung \ref{fig:haar} sehen, wie die Nase viel heller
	als die Region bei den Augen ist. Die Selektion von den besten Merkmalen wird
	durch den Adaboost Algorithmus berechnet. 
	
	\begin{figure}[H]
		\centering
		\includegraphics[scale=0.8]{\ordnerfigures haar.png}
		\caption{Haar}
		\label{fig:haar}
	\end{figure}
	
	Mit dieser Absicht wenden wir alle Funktionen auf alle Trainingsbilder an. Für
	jedes Merkmal wird der beste Schwellenwert ermittelt, der die Gesichter in
	positive und negative klassifiziert.\\
	



\subsubsection{Umsetzung in Code}


	
	In Listing \ref{detect} ist ein Code Abschnitt der Gesichtserkennung
	dargestellt. \\
	
\begin{lstlisting}[language=Python, caption=Kern Code für
Gesichtsdetektion,label=detect]
face_cascade = 
cv2.CascadeClassifier('haarcascade_frontalface_default.xml')
	
image = cv2.imread('gesicht.jpg')
	
gray = cv2.cvtColor(image,cv2.COLOR_BGR2GRAY)
	
faces = face_cascade.detectMultiScale(gray,1.1,4)
	
for(x, y, w, h) in faces:
	
cv2.rectangle(image, (x, y), (x+w, y+h), (255,0,0), 2) 
	
\end{lstlisting}
	
	
	Es wurde die trainierte Klassifikator-XML-Datei (haarcascade\textunderscore
	frontalface\textunderscore default.xml) die sich im GitHub-Repository von OpenCV
	befindet, verwendet.
	In den Zeilen \textbf{1} und \textbf{2} wurde sie geladen und in die Variable
	face\textunderscore cascade gespeichert.
	In den Zeile \textbf{4} wird das Bild von den Funktion \emph{imread} geladen.
	Die Erkennung funktioniert nur bei Graustufenbildern. Daher wurde das Farbbild
	in Graustufen umgewandelt, wie in der Zeile \textbf{6} ersichtlich.
	Der Funktion \textit{ detectMultiScale } in der Zeile \textbf{8} erkennt die
	Gesichter im Bild. Diese Funktion ist sehr wichtig und braucht 3 Argumente – das
	Eingabebild, der Skalierungsfaktor (scaleFactor) und die Zahl
	\textit{minNeighbours}. 
	\\
	\textit{scaleFactor} gibt die an, um wie viel das Bild vergrößert bzw.
	verkleinert wird. \\
	\textit{minNeighbours} gibt an, wie viele Nachbarrechtecke jedes Rechteck haben
	muss. Dieser Parameter bestimmt die Qualität der erkannten Gesichter: Ein
	höherer Wert führt zu weniger Erkennungen, jedoch zu einer höheren Qualität.
	Es ist bei uns nicht so entscheidend viele Gesichter zu erkennen, sondern die
	Qualität. Es wird der Wert 4 benutzt.
	\\
	In Zeile \textbf{10} wird durch die Gesichter \textit{faces} iteriert. \\
	\textit{x} und \textit{y} entsprechen den Koordinaten vom Bild, \textit{w},
	\textit{h} bezeichnen die Breite und Höhe \textit{width}, \textit{height}.
	\\
	Schließlich wird in Zeile \textbf{12} mit den Funktion \textit{cv2.rectangle}
	ein Rahmen auf dem Gesicht gezeichnet. \\
	Die Funktion bekommt diese Parameter: \textit{image} für die Bildeingabe,
	\textit{x, y} beschreiben den Startpunkt, \textit{w, h} den Endpunkt, die Farbe
	und die Dicke der Rahmen werden in den 2 letzten Parameter angegeben.
	\\
	Dieser Code wurde so umgeändert, dass er in der Lage ist, durch mehrere
	\textit{Predictors} bzw. Klassifikatoren, die Genauigkeit der Gesichtserkennung
	zu erhöhen. \\
	
	Auf Abbildung \ref{fig:aron} sieht man deutlich ein erkanntes Gesicht.
	\begin{figure}[H]
		\centering
		\includegraphics[scale=0.6]{\ordnerfigures aronfd.png}
		\caption{Output: Box auf Gesicht}
		\label{fig:aron}
	\end{figure}



\subsection{Normalisierung}

\label{normal}

	
	In dieser Kapitel wird ein sehr wichtiger Bereich, die sogenannte Normalisierung
	eines Bildes, genau erklärt. 
	
	\subsubsection{Was ist Normalisierung?}
	
	Normalisierung in \textit{Computer Vision} ist der Prozess der Bereitstellung
	von Daten, bei dem "falsche" Daten, die die Genauigkeit unseres Systems
	beschädigen können, korrigiert werden\\ 
	Normalisierung heißt in unserem Fall Ausrichtung. 
	Die Ausrichtung der Gesichter, die im Bild "falsch" positioniert sind. \\
	Also entspricht Normalisierung dem Prozess in dem man zuerst die geometrische
	Struktur von Gesichtern in digitalen Bildern identifiziert und dann versucht
	eine maßgebliche Ausrichtung des Gesichts basierend auf Skalierung und Rotation
	zu erhalten.
	\subsubsection{Methoden für Normalisierung}
	Es gibt viele Methoden mit denen man Bilder normalisieren könnte. 
	Einige von denen basieren auf pre-definierten 3D-Modellen und transformieren
	dadurch die Eingabebilder, sodass die Gesichtsschlüsselpunkte der Eingabebild
	mit denen von dem 3D Modell übereinstimmen.
	\\
	Die in dieser Diplomarbeit verwendete Methode, eine eher einfachere, verlässt
	sich nur auf die Gesichtspunkte selbst, um eine normalisierte Darstellung des
	Gesichts durch Affine- und Ähnlichkeits-Transformation zu erhalten.
	Diese Methode wurde deswegen implementiert, weil sie sehr effizient ist.
	
	\subsubsection{Warum Normalisierung?}
	Der Grund warum in dieser Arbeit Normalisierung von Daten durchgeführt wird, ist
	dass viele Gesichtserkennungsalgorithmen, einschließlich unserem, von der
	Anwendung dieser Ausrichtung sehr profitieren. 
	Es wird dadurch die Präzision der Gesichtserkennung erhöht. 
	
	
	\subsubsection{Implementation}
	
	Die hier verwendete Methode für die Normalisierung, wie oben kurz erwähnt wurde,
	wendet
	Ähnlichkeitstransformation bei zwei Paaren entsprechender Punkte an.
	Die Punkten sind die von Dlib extrahierten Gesichtsmerkmale, siehe Kapitel
	\ref{gspex}.
	\\
	
	OpenCV benötigt in diesem Fall 3 Punkte zur Berechnung der Ähnlichkeitsmatrix.
	Wir nehmen somit als dritten Punkt, den dritten Punkt eines gleichseitigen Dreiecks
	mit diesen beiden gegebenen Punkten an.
	\\
	
	Was eine Ähnlichkeitstransformation ist, lässt sich mathematisch beschreiben. 
	Eine Ähnlichkeitstransformation ist eine Transformation wie z.b.: Reflexion,
	Rotation oder Translation. 
	\\
	
	Wenn eine Figur durch eine Ähnlichkeitstransformation transformiert wird, wird
	ein Bild erstellt, das der ursprünglichen Figur ähnlich ist. Mit anderen Worten,
	zwei Figuren sind ähnlich, wenn eine Ähnlichkeitstransformation die erste Figur
	zur zweiten Figur transformiert.\\
	
	Ein solches Prinzip ist auch in unsere Methode eingesetzt worden. Es wird eine
	zweites, dem ersten Bild ähnliches Bild, erstellt. \\
	
\begin{lstlisting}[caption=Implementation
	Normalisierung,language=python,label=norm]
	
def similarityTransformMat(initialPoints, destinationPoints):
...
tform = cv2.estimateAffinePartial2D(np.array([initialPoints]),
	np.array([destinationPoints]))
return tform[0]
	
\end{lstlisting}
	
	Im Code Abschnitt \ref{norm} wird die Methode gezeigt, die als Parameter die
	Anfangs- und die Zielpunkte nimmt und aus den beiden jeweils einen dritten Punkt
	berechnet.
	\\
	
	Sie werden dann in Arrays gespeichert und aus denen wird die
	Ähnlichkeitstransform von den in OpenCV eingebetteter Funktion
	\textit{cv2.estimateAffinePartial2D} berechnet.
	\\
	
	Ausgabe ist eine 2D affine Transformation, also eine 2x3 Matrix, oder eine leere
	Matrix, wenn die Transformation nicht geschätzt werden konnte.\\
	"Die Funktion schätzt eine optimale affine 2D-Transformation mit 4
	Freiheitsgraden, die auf Kombinationen aus Translation, Rotation und
	gleichmäßiger Skalierung beschränkt sind. Verwendet den ausgewählten Algorithmus
	für eine robuste Schätzung."\cite{opencv_library}
	\\
	
	Die nächste Funktion macht die Ausrichtung des Gesicht im Bild. 
	Es bekommt ein, von der Funktion \textit{cv2.imread} eingelesenes Bild, die
	gewünschte Größe, und die Dlib Gesichtsmerkmale als Parameter. 
	
	\begin{lstlisting}[caption=Gesichtsausrichtung,label=align,language=python]
	def faceAlign(image, size, faceLandmarks):
	(h, w) = size
	initialPoints = []
	destinationPoints = []
	
	initialPoints = [faceLandmarks[36], faceLandmarks[45]]
	
	destinationPoints = [(np.int(0.3*w), np.int(h/3)), (np.int(0.7*w),
	np.int(h/3))]
	
	similarityTransform = similarityTransformMat(initialPoints, destinationPoints)
	
	faceAligned = np.zeros((image.shape), dtype=image.dtype)
	
	faceAligned = cv2.warpAffine(image, similarityTransform, (w, h))
	
	return faceAligned
	
	\end{lstlisting}
	
	Im Code Abschnitt \ref{align} in Zeile \textbf{6} wird die Position der linken
	Ecke des linken Auges und der rechten Ecke des rechten Auges von dem Eingabebild
	genommen. \\
	In Zeile \textbf{8} wird die Position der linken Ecke des linken Auges und der
	rechten Ecke des rechten Auges im ausgerichtete Bild berechnet. \\
	In Zeile \textbf{10} und \textbf{11} wird die Ähnlichkeitstransformation durch
	die vorher erstellten Funktion \textit{similarityTransformMat} berechnet.\\
	In Zeile \textbf{12} wird das ausgerichtete Gesicht in einem Tupel gespeichert.
	\\
	Schließlich wird die Ähnlichkeitstransformation durch die Methode
	\textit{cv2.warpAffine} angewendet.\\
	Diese Methode bekommt das Tupel vom Bild, \textit{similarityTransform}, und die
	Größen vom Bild als Parameter. Das Ergebnis wird dann von den Funktion
	zurückgegeben. (Zeile \textbf{16}) 
	\\
	
	Eine affine Transformation ist jede Transformation, die die Kollinearität (d. H.
	Alle auf einer Linie liegenden Punkte, die anfänglich nach der Transformation
	noch auf einer Linie liegen) und die Entfernungsverhältnisse (z. B. der
	Mittelpunkt eines Liniensegments bleibt der Mittelpunkt nach der Transformation)
	bewahrt. 
	Eine affine Transformation wird auch als Affinität bezeichnet.\cite{affine}
	\\ 
	Zusätzlich wurde noch eine Glättungsfilter angewendet. Dies geschieht durch die
	Funktion \textit{cv.2filter2d}, die ein Bild mit dem Kernel faltet. \\
	Sie bekommt folgende Parameter: 
	\textit{output} - Zielbild.\\
	\textit{kernel} - Faltungskern (oder vielmehr ein Korrelationskern), eine
	einkanalige Gleitkommamatrix.\\
	\textit{ddepth} - gewünschte Tiefe des Zielbildes. \\
	
	Der Kernel ist auf Code abschnitt \ref{smooth} genau sichtbar. 
	
	\begin{lstlisting}[language=python,caption=Glättungsfilter,label=smooth]
	kernel = np.ones((5,5),np.float32)/25
	
	output = cv2.filter2d(output,-1, kernel)
	\end{lstlisting}
	
	\begin{figure}[H]
		\centering
		\subfloat[Nicht normalisiert]{{\includegraphics[scale=0.5]{\ordnerfigures
					NewAron4.jpg}} }%
		\qquad
		\subfloat[Normalisiert]{{\includegraphics[scale=0.315]{\ordnerfigures
					alignedAronCropped.jpg}} }%
		\caption{Unterschied: normalisiert, nicht normalisiert}
		\label{fig:norme}
		
	\end{figure}
	
	\begin{figure}[H]
		\centering
		\subfloat[Nicht normalisiert]{ {\includegraphics[scale=0.7]{\ordnerfigures
					Neworens1.jpg} }}%
		\qquad
		\subfloat[Normalisiert]{ {\includegraphics[scale=0.315]{\ordnerfigures
					alignedorens1Cropped.jpg} }}%
		\caption{Unterschied: normalisiert, nicht normalisiert}
		\label{fig:norm}
	\end{figure}
	
	
	Auf Abbildungen \ref{fig:norm} und \ref{fig:norme} sieht man ganz klar die
	Unterschiede die die Normalisierung in Bildern verursacht. 

	
\subsection{Zuschneiden von Gesichter}

Nachdem Gesichter gefunden werden, muss man daraus eigene Bilder machen, die dasExtrahieren der Schlüsselpunkte erleichtern.\\  

\begin{lstlisting}[language=python,caption=Code Abschnitt: Gesicht
Zuschneiden,label=crop]
if nrFace > 0:
for face in faces:
for(x, y, w, h) in faces:
r = max(w, h) /2 
centerx = x + w /2 
centery = y + h /2
nx = int(centerx - r) 
ny = int(centery - r) 
nr = int(r * 2) 
faceimg = image[ny:ny+nr, nx:nx+nr] 
filenam = input("Give new filename for cropped photo: \n")
image2 = cv2.imwrite(filenam,faceimg)
elif nrFace <= 0:
print("no faces found")

\end{lstlisting}

In Listing \ref{crop} sieht man folgendes:\\

\textit{nrFace} bestimmt die Anzahl der Gesichter, die von den
Kaskadenklassifikator gefunden wurden. 
Nur wenn dieser Wert größer als 0 ist soll das Programm weiterlaufen.  
Es wird durch jedes Gesicht iteriert.
\\

Zeilen \textbf{1-6} berechnen das Zentrum vom neuen Bild und in Zeile \textbf{7}
wird ein neues Bild mit den berechneten Parametern angelegt.\\
Mit imwrite wird das Bild gespeichert (Zeile \textbf{12}). 
Auf Abb.\ref{fig:aroncropped} sieht man deutlich nur das Gesicht.
\\

\begin{figure}[H]
	\centering
	\includegraphics[scale=0.6]{\ordnerfigures aroncropped.jpg}
	\caption{Output: Zugeschnittenes Bild}
	\label{fig:aroncropped}
\end{figure}

	
\subsection{Gesichtsschlüsselpunkte Extraktion}
\label{gspex}

Nun kommt ist es zu einem sehr wichtigen Punkt meines Teils der Diplomarbeit, die Extraktion Gesichtsschlüsselpunkte. 
Dazu wurden die Gesichtsmerkmale \textit{„Facial Landmarks“} von dlib verwendet.

Die Gesichtsmarkierungen werden verwendet, um Bereiche des Gesichts zu finden
und darzustellen, wie zum Beispiel: die Augen, die Augenbrauen, die Nase, den
Mund und den Kiefer. Wie macht man das? Wie erkennt man Gesichtsmerkmale? \\Das ist eine Teilmenge des Problems der Formvorhersage. Bei einem vorgegebenen Eingabebild (und
normalerweise einem ROI\footnote{Region of interest, Region von Interesse}, die
das interessierende Objekt angibt) versucht ein Formvorhersager, wichtige Punkte
entlang der Form zu lokalisieren. 
Dieser Formvorhersager, der im Dlib integriert ist wurde von Kazemi and Sullivan
in ihrem Paper: \textit{One Millisecond Face Alignment with an Ensemble of
	Regression Trees} entwickelt.\cite{Kazemi2014OneMF} 
Dieser Methode werden viele Trainingsdaten zur Verfügung gestellt, damit sie ein Kombination von Regressionskurven trainiert, um die Positionen der \textit{Facial Landmarks} direkt aus den Pixelintensitäten selbst zu berechnen.\cite{Kazemi2014OneMF} \\

Dadurch werden unsere gewünschten 68 Gesichtsmarkierungen mit hohen
Vorhersagbarkeit extrahiert und als x, y Koordinaten weitergegeben. 
Zusätzlich ist es auch so, dass diese Koordinaten nicht von den
Dimensionen des Bildes abhängig sind. 
Die Indizes der 68 Koordinaten sind in der Abb.\ref{fig:landmarks} dargestellt:
\\

\begin{figure}[H]
	\includegraphics[scale=0.5]{\ordnerfigures landmarks.png}
	\centering
	\caption{Gesichtsschlüsselpunkte \cite{Kazemi2014OneMF}}
	\label{fig:landmarks}
\end{figure}

Nachdem der Prediktor geladen wird, speichern wir die Gesichtsschlüsselpunkte in einem Formobjekt
mit 68(x,y) Koordinaten. \\
Es wird hier nicht die \textit{resize} Funktion verwendet, weil das Bild an Qualität verliert
und es rechenintensiver ist, je größer das Eingabebild wird. \\
Jede Koordinate läuft in einer Schleife durch und entspricht dem spezifischen
Gesichtsmerkmal im Bild. Auf Abbildung \ref{fig:data} sieht man die extrahierten Daten von einem Gesicht. \\
\begin{figure}[H]
	\includegraphics[scale=0.8]{\ordnerfigures pikat.png}
	\centering
	\caption{Gesichtsdaten}
	\label{fig:data}
\end{figure}


Die Merkmale werden in einer numpy Array gespeichert für die
Speicherung in der Datenbank. \\



\section{Herausforderungen, Probleme und deren Lösung }



Die größte Herausforderung lag bei der Planung der Software und den Methoden
die zum Extrahieren von den Gesichtsschlüsselpunkten dienten.\\
Es hat mich viel Zeit gekostet, bis ich eine geeignete Lösung gefunden habe und
das hat mir viel Stress gemacht. \\
Eine weitere Herausforderung war das Verknüpfen von den Entwicklungsumgebungen
und die Kooperation zwischen den Teammitgliedern. \\
Die verwendete Systeme waren all zu unterschiedlich und es konnte keine
Standardisierung zwischen ihnen gefunden werden. Also ist viel Zeit beim
Installieren und Konfigurieren investiert worden. Ein Grund dafür ist die
mangelnde Erfahrung mit den verwendeten Technologien. 
\\

Viele Sachen, wie z.b.: die Einteilung der Arbeit, die genaue Spezifikationen,
die sehr grobe Planung, war am Beginn auch wegen den
Kommunikationslücken unklar. 
\\

Als das Projekt weiter fortschritt, hat sich der Bedarf an Genauigkeit und Präzision
auch enorm gesteigert.
Es hat immer wieder Fälle beim Finden von Gesichter gegeben, in denen die fertig gestellten neuronale Netze
nicht so gut funktioniert haben . 
\\

Es war keine gute Idee sich nur von einem Prädiktor abhängig
zu machen. 
Es war deutlich dass, die Einsetzung von mehreren neuronalen Netzen Erhöhung der Treffsicherheit ergeben hätte.
\\

Eine einfache Normalisierung genügte auch nicht die Umwandlung der Bilder in
Graustufe), sondern es war eine komplexere Lösung dafür notwendig.
\\
Bis das funktioniert hat, hat es viel Zeit und Mühe gekostet. 
Dafür waren auch relativ fortgeschrittene mathematische Kenntnisse nötig.
Dabei habe ich viel Try and Error angewendet.
\\
 
Es hat auch Schwierigkeiten beim Qualitätsmanagement gegeben. Die
Erfahrung fehlte, somit waren gewisse Maßstäbe und Standarts auch nicht bekannt. 



\subsection{Lösungen}
Während der Arbeit habe ich viel recherchiert und mich genau über alles Mögliche
informiert. 
\\

Es wurde viel herumprobiert und experimentiert. Es wurde auch sehr viel
getestet, damit die Ziele auch qualitativ erfüllt wurden. 
\\
Die Kommunikation hat sich steigendem mit Bedarf über die Zeit stark verbessert und
dadurch sind auch die Unklarheiten abgeklärt worden. 


\section{Qualitätssicherung, Controlling}


Die Qualität wurde durch verschiedene Methoden gesichert. Eine von denen war die
Methode 5xWarum. 
\\
 
„Fünf warum“ ist eine iterative Methode, die Fragen als Basis hat und die
Beziehungen zwischen Fragen und Problemen. \\
Es geht hier um die Verschachtelung der Ursachen und das Herausfinden von ihnen durch iterative Fragetechnik, da viele Probleme nicht nur eine einzige Ursache haben. 
Die Methode ruft jedes Mal eine andere Folge von Fragen auf.\cite{fmea}
\\

Die aufgetauchten Probleme haben viele Ursachen, die nicht mit dem ersten Blick
sichtbar sind. 5x warum hilft durch diese verschachtelten Ursachen das
grundlegende Problem zu entdecken. \\

Eine Problemstellung: „Segmentation fault“ beim Finden von Keypoints mit
FAST\footnote{Features from Accelerated Segment Test}.“ 


\begin{enumerate}
	
	
		\item  \textbf{Was ist überhaupt ein „Segmentation Fault“? }
	Ein Segmentierungsfehler tritt auf, wenn ein Programm versucht, auf einen
	Speicherort zuzugreifen, auf den es nicht zugreifen darf, oder wenn versucht
	wird, auf einen Speicherort auf nicht zulässige Weise zuzugreifen (z. B. beim
	Versuch, an einen schreibgeschützten Speicherort zu schreiben, oder einen Teil
	des Betriebssystems zu überschreiben).
	\item \textbf{Warum passiert das, warum versucht mein Program auf einen Speicherort
	zuzugreifen, auf den es nicht zugreifen darf?}\\
	Problem ergibt sich in dieser Zeile: 
	“kp = fast.detect(image, None)”
	
	Wahrscheinlich durften die Daten die fast.detect zurückgibt nicht in kp gespeichert
	werden. 
	
	ODER\\
	
	Das Paket in dem die FAST Algorithmus drinnen ist ist nicht (richtig)
	installiert worden.
	Warum? Alle nötigen Pakete wurden in einer gesamten Installation geholt und
	FAST war nicht da. 
	
	\item	\textbf{Warum dürfen die Daten die fast.detect zurückgibt nicht in kp gespeichert
	werden ? }
	
	Die Daten die fast.detect zurückgibt, dürfen nicht in kp gespeichert werden, weil
	kp kein Array ist. 
	
	\item	\textbf{Warum ist kp kein Array? }
	
	Kp ist kein array, weil man es in Python  manuell angeben muss.
	
	\item \textbf{Warum wurde es nicht manuell angegeben? }
	
	Es wurde nicht manuell gegeben weil, die Methode fast.detect nicht gut
	recherchiert wurde. Man sollte mehr darüber in die Dokumentation nachschauen.
	
	\textbf{Konklusion}\\
	Durch die 5x Warum Methode ist es zu den Ergebnis gekommen: Es musste ja mehr
	untersucht werden bevor man eine solche Methode implementiert. Die „5xWarum“
	Methode hat dabei geholfen, dass die eigentliche Ursache des Problems
	herausgefunden wurde.\\
	Die nächste Schritte sind: entweder eine andere Methode, die schon installiert
	ist, verwenden oder die benötigten Pakete für FAST manuell installieren. 
	
	
	
\end{enumerate}


\section{Ergebnisse}

	
Ich habe ungefähr 180 Stunden Zeit ins gesamt für die Diplomarbeit investiert und folgendes erreicht.\\

\begin{itemize}
	\item Gesichtserkennung und das Zuschneiden von Gesichter wurden fertig
	implementiert.
	\item Bilder sind normalisiert worden.\\
	\item Gesichtsschlüsselpunkte wurden extrahiert.\\
	
\end{itemize}


\section{Evaluierung und Resümee}

	In diesem Kapitel geht es um die Reflexion, Evaluierung der Ergebnisse. Was von der Planung abgewichen ist usw. 

\subsection{Planung vs. Realisierung}


Es wurde am Anfang so geplant, dass die Klassifikatoren und die neuronale Netze alle selbst gemacht werden müssten. 
Über die Zeit wurde mir klar, dass so was zu kompliziert und fast unmöglich war.
Die Genauigkeit der Gesichtserkennungsalgorithmen könnte nie erreicht werden, wenn es vom Beginn erarbeitet wurde. Geplant waren auch gewisse Ziele, die über die Diplomarbeit-Mitglieder falsch verteilt worden waren. Z.b.: Ein Ziel von mir war die Vektorumwandlung, die später so geschätzt wurde, dass es eigentlich zu Rei's Teil gehörte. 
	
	

\subsection{Ergebnisse}

	Im technischen Aspekt habe ich viel gelernt.
	Diese Diplomarbeit hat viele Kenntnisse mit sich gebracht. Zahlreiche neue Bibliotheken wurden verwendet. 
	Ein ganz unbekanntes Framework ist kennen gelernt worden. 
	Ich habe viel Erfahrung mit Computer Vision bekommen.
	Da meine Aufgaben zum Teil sehr anspruchsvoll waren, habe ich viel Zeit und Mühe für die Realisierung investiert.
	Soft Skills wurden auch dazugewonnen. Ich habe erlernt wie wichtig Teamwork und Kommunikation ist. 
	Wenn ich die ganze Diplomarbeit noch einmal machen würde, wäre vieles anders. 
	  

